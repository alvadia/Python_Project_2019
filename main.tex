
\documentclass[final,12pt]{article}
\usepackage[utf8]{inputenc}
\usepackage[english,russian]{babel}
\usepackage{vmargin}
\usepackage{verbatim}
\setpapersize{A4}
\setmarginsrb{3cm}{2cm}{1.5cm}{2cm}{0pt}{0mm}{0pt}{13mm}
\usepackage{setspace}
\usepackage{mathtools}
\usepackage{amsfonts}
\newcommand{\bigO}{\mathcal{O}}
\onehalfspacing
\begin{document}
\thispagestyle{empty}

\begin{center}

{\scshape Московский государственный университет им.М.В.~Ломоносова}\\
Факультет вычислительной математики и кибернетики\\


\vspace{7cm}

\textbf{{\Large Отчет по заданию практикума}}

\vspace{1cm}

{\Huge\bfseries
<<Система контроля ассортимента книжного магазина>>}
\end{center}

\vspace{3cm}

\begin{flushright}
 
 \vspace{2cm} 

{{\normalsize Зизов Вадим, 425 группа }}
\end{flushright}
  
  \vspace{5cm}
  
\begin{center}
Москва, 2019
\end{center}

\enlargethispage{4\baselineskip}

\newpage

\tableofcontents

\newpage

\section{Уточнение постановки задачи}

Книжный магазин осуществляет продажу широкого ассортимента книг нескольких издательств. Книги различаются по тематике и категории читателей (детская литература, учебники, научная литература по отдельным областям знаний, литература на иностранных языках, научная фантастика, фэнтези и т.п.).

Компьютерная система контроля ассортимента хранит данные о наличии и количестве экземпляров книг в магазине, при этом для каждой книги хранятся сведения о ее авторе, названии, издательстве, годе издания, количестве страниц, тематике и категории, цене и розничной наценке, рейтинге спроса. Для новых книг розничная наценка на некоторый фиксированный период устанавливается больше обычной. 

В течение каждого рабочего дня система фиксирует заказы на книги (заказы записываются в магазине, а также поступают по телефону и электронной почте). Заказ включает фамилию покупателя, его номер телефона или электронный адрес, а также перечень заказываемых авторов, с книгами и указанием их количества. В заказе может быть указан только автор книги, с требованием новой книги, такая заявка выполняется с последней изданной книгой данного автора. Если требуемая в заказе книга имеется в магазине, то она откладывается для покупателя, и делается соответствующая запись о продаже. В противном случае система отмечает заявку как невыполненную, а книгу как недоставленную. Рейтинг спроса каждой книги рассчитывается по числу ее заказов, по числу фактических продаж, и по общей сумме продаж. 

Система отслеживает фактическое количество экземпляров каждой книги в магазине. Если оно становится меньше определенного порога, то составляется заявка в издетельство на доставку в магазин дополнительных экземпляров этой книги. Более формально, для любой книги, которая не доставляется и количество экземпляров которой в магазине на конец рабочего дня меньше фиксированного параметра моделирования, составляется заявка с указанием числа желаемых экземпляров, в зависимости от рейтинга книги. Заявки выполняются в течение нескольких дней.

Основная функция системы управления ассортиментом - автоматизация обработки заказов на книги и составления заявок в издательства. Для тестирования работы системы смоделирован поток поступающих заказов. Период моделирования переменный, шаг моделирования - один рабочий день.

Поток заказов на книги смоделирован статистически: все составляющие заказа подбираются случайным образом, но при этом новые книги заказываются чаще. Плотность потока заказов зависит от разнообразия ассортимента книг в магазине. Фактический срок доставки книг в магазин (т.е. срок выполнения заявки в издательство) также моделируется с помощью случайной величины.

В параметры моделирования работы книжного магазина включено число \(N\), начальный ассортимент книг в магазине, дипазоны разброса указанных случайных величин, процент обычной розничной наценки и наценки на новые книги. В ходе моделирования доступна информация об ассортименте магазина, о поступивших и обработанных заказах, а также о выполненных заявках в издательство. По окончании моделирования выводится статистическая информация о работе магазина.

\section{Диаграмма основных классов}
\includegraphics[width=1.0\textwidth]{PyDiaObjects.png}
\section{Текстовые спецификации интерфейса}
\verbatiminput{specification.txt}
\section{Диаграмма объектов}
\includegraphics[width=1.0\textwidth]{objects.png}
\section{Инструментальные средства}
Язык разработки: Python 3

Инструментальная среда:

Jupyter, Spyder

Используемые библиотеки:

collections, random, PyQt5

\section{Файловая структура}
\verbatiminput{file_system.txt}
\section{Пользовательский интерфейс}
Пользовательский интерфейс представляет собой единое окно. В начале сессии предлагается выбрать настройки, которые все имеют некоторое стартовое значение. На экране отображаются краткие характеристики параметров, и панели ввода для их изменения.

\includegraphics[width=1.0\textwidth]{start_setup.png}

После начала моделирования каждый шаг моделируется отдельно, нажатием клавишы Enter либо соответствующей кнопки окна. Основная информация выводится в трёх таблицах, соответствующих книгам, заявкам в издательства и покупателям соответственно.

\includegraphics[width=1.0\textwidth]{step_60.png}

После окончания моделирования выводится статистическая информация, отсортированная в каждом списке по принесённой магазину прибыли.

\includegraphics[width=1.0\textwidth]{final_20.png}
\end{document}